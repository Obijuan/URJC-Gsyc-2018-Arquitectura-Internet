% !TeX TXS-program:compile = txs:///pdflatex/[--shell-escape]

\documentclass[a4paper]{article}
\usepackage{fullpage}
\usepackage[spanish]{babel}

\usepackage[default]{lato}  %% Option 'sfdefault' only if the base font of the document is to be sans serif
\usepackage[scaled=.9]{noto-mono} %% Change default mono font
\usepackage[T1]{fontenc}
\usepackage[utf8]{inputenc}

\usepackage{graphicx}
\usepackage[dvipsnames]{xcolor}
\usepackage{colortbl}

% Special fonts and glyphs
% \usepackage{palatino}
\usepackage{fontawesome}
\usepackage{adforn}

% setlists (itemize, enumerate) with customized icons
% easy restart of number sequence in enumerate
% http://ctan.org/pkg/enumitem
\usepackage[shortlabels]{enumitem}

\newcommand{\usageitem}[1]{%
  \item[%
    {\makebox[2em]{\strut #1}}%
  ]
}

%% Ajustar interlineado entre párrafos a una línea en blanco
\setlength{\parskip}{2mm}

% Roman numbers for section
\renewcommand{\thesection}{\Roman{section}} 
\renewcommand{\thesubsection}{\thesection.\Arabic{subsection}}

\usepackage{url}
%%% PDF file metadata

%%% Customized colors for this document
\definecolor{darkred}{rgb}{0.5,0,0}
\definecolor{darkgreen}{rgb}{0,0.5,0}
\definecolor{darkblue}{rgb}{0,0,0.5}

% Colored boxes for information corners
\usepackage{tcolorbox}% http://ctan.org/pkg/tcolorbox

% Caja azul con título para información, explicaciones extra y detalles
% Icons: \faInfoCircle \faBook \faBookmark
\newtcolorbox{mybluebox}[1]{colback=darkblue!5!white,colframe=darkblue!75!black,
                            fonttitle=\normalfont,title=#1}
                            
% Caja verde para comentarios adicionales
% Icons: \faComment \faCrosshairs
\newtcolorbox{mygreenbox}[1]{colback=darkgreen!5!white,colframe=darkgreen!75!black,
                             fonttitle=\normalfont,title=#1}
                             
% Caja roja con título para posibles problemas explicaciones extra y detalles
% Icons: \faExclamationCircle \faBomb
\newtcolorbox{myredbox}[1]{colback=darkred!5!white,colframe=darkred!75!black,
                           fonttitle=\normalfont,title=#1}

\usepackage{minted}
% \usepackage{tcolorbox}
% \tcbuselibrary{minted,skins}
% 
% \newtcblisting{bashcode}{
%   listing engine=minted,
%   colback=bashcodebg,
%   colframe=black!70,
%   listing only,
%   minted style=colorful,
%   minted language=bash,
%   minted options={linenos=true,texcl=true},
%   left=1mm,
% }
% \definecolor{bashcodebg}{rgb}{0.85,0.85,0.85}
% 
% \newtcblisting{pycode}{
%   listing engine=minted,
%   colback=pycodebg,
%   colframe=black!70,
%   listing only,
%   minted style=colorful,
%   minted language=python,
%   minted options={linenos=true,texcl=true},
%   left=1mm,
% }
% \definecolor{pycodebg}{rgb}{0.85,0.85,0.85}

\usepackage[pdftex]{hyperref}

\definecolor{darkred}{rgb}{0.5,0,0}
\definecolor{darkgreen}{rgb}{0,0.5,0}
\definecolor{darkblue}{rgb}{0,0,0.5}

\hypersetup{
    pdftoolbar=true,	%Shows Adobe Acrobat toolbar
    pdfmenubar=true,	%Shows Adobe Acrobat menu
    pdftitle={AI - Bloque III - Práctica - 2020},
    pdfauthor={Juan González, Felipe Ortega},
    pdfcreator={GSyC, ETSIT},
    pdfproducer={PDFLaTeX},
    pdfsubject={AI - Evaluación - Práctica - 2020},
    pdfnewwindow=true,  %links open in new window
    colorlinks=true,    %false: boxed links; true: coloured links
    linkcolor=darkred,
    citecolor=darkgreen,
    urlcolor=darkblue
}

\begin{document}
	
\includegraphics[width=2.5cm]{gsyc-logo-small.png} \hspace{8cm} \includegraphics[width=4cm]{URJClogo.png}

%--- Heading metadata
\begin{center}
 \LARGE{\textbf{Evaluación Bloque III (Práctica)}}
 \smallskip
 
 \large{\textbf{Arquitectura de Internet}}
 
 \large{Grado en Ing. Sistemas Audiovisuales y Multimedia}
 
 \large{ETSIT, URJC}
 \smallskip
 
 \large{\underline{\today}}
\end{center}
\bigskip

\hrule
\smallskip

\setlist[itemize]{leftmargin=12mm}

\begin{itemize}[leftmargin=0.8cm,itemsep=6pt,topsep=6pt]
  \usageitem{\faSend} \textbf{Envío}: El envío se realizará a través del espacio de entrega
  habilitado en el apartado \textit{Evaluación} de la página de la asignatura en Aula Virtual.
	
  \usageitem{\faWrench} \textbf{Herramientas software}: Utiliza NetGUI, Wireshark así como las herramientas
  Linux que se han ido introduciendo en las Prácticas 3 a 5 para contestar las preguntas de
  cada apartado. 
  
  \usageitem{\faFileO} \textbf{Formato}: Puedes enviar tus respuestas en un fichero de procesador 
  de textos (LibreOffice u Open-Office) o en formato PDF. También es válido componer un 
  documento con una herramienta de notas para tables, siempre que el envío se realice 
  \textbf{en formato PDF}.
  
  \usageitem{\faExclamationCircle} \textbf{Advertencia}: Incluye claramente en el documento
  de tu respuesta tus datos personales, así como \textbf{toda la información solicitada} en
  cada pregunta (pantallazos, comandos, justificaciones, etc.). De lo contrario, la respuesta
  no puntuará.
  
  \usageitem{\faCalendar} \textbf{Fecha tope de entrega}: \textbf{\underline{Domingo, 31 de mayo de 2020}}.
\end{itemize}

\medskip
\hrule
%--- End of heading metadata

\bigskip

\section{Enrutamiento IP}
\label{sec:ip-routing}

Descomprime el escenario disponible en el fichero \texttt{escenario-test.zip} y ábrelo con NetGUI. 
Arranca todas las máquinas.

\begin{enumerate}
	\item  Obtén las tablas de enrutamiento de \textbf{\underline{todas las máquinas}} que estén 
	configuradas e indica el comando que hay que usar. Adjunta pantallazos de los resultados en el 
	terminal en todas las máquinas.
	
	\item Comprueba si \texttt{pc2} y \texttt{pc4} pueden intercambiar datagramas. Indica el comando
	que has usado para ello, la salida que produce y por qué deduces que hay conectividad.
	
	\item Los paquetes enviados de \texttt{pc2} a \texttt{pc4}, ¿qué ruta siguen? Justifícalo, indicando
	los comandos que hay que ejecutar para saberlo, y las salidas que producen.
	
	\item Los paquetes enviados de \texttt{pc4} a \texttt{pc2}, ¿qué ruta siguen? Justifícalo, indicando 
	los comandos que hay que ejecutar para saberlo, y las salidas que producen.
	
	\item Configura la IP de \texttt{pc1} para que haya conectividad con \texttt{pc2}. Los paquetes de
	\texttt{pc1} a \texttt{pc2} deben seguir la ruta \texttt{pc1} $\rightarrow$ \texttt{r2} $\rightarrow$
	\texttt{pc2}. Sin cambiar ninguna tabla de enrutamiento del resto de máquinas, ¿qué ruta siguen
	los paquetes que van de \texttt{pc2} a \texttt{pc1}?
	
	\item Apaga \texttt{r1}, \texttt{r2}, \texttt{r3} y \texttt{r4}, por ese orden. Configura las máquinas
	restantes para que haya conectividad entre todas ellas. Muestra la información del comando 
	\texttt{traceroute} desde \texttt{pc1} al resto de máquinas, y justifica por qué tu configuración está 
	funcionando. Guarda esta configuración de forma persistente.
\end{enumerate}

%%%---------------------------------------------- 

\section{ARP}
\label{sec:arp}

\begin{enumerate}[resume]
	\item Partimos del escenario que ya tenías configurado en la \textbf{pregunta  6}. Asegúrate que 
	hay conectividad entre \texttt{pc2} y \texttt{pc4} a través de \texttt{r5}. Cierra todas las máquinas 
	y arranca sólo \texttt{pc2}, \texttt{pc4} y \texttt{r5}. Completa la siguiente tabla:
\end{enumerate}

	\bgroup
	\def\arraystretch{1.5}%  1 is the default, change whatever you need
	\begin{tabular}{|l|p{5cm}|p{4cm}|}
		\hline 
		\textbf{Nombre máquina} & \textbf{Dirección Ethernet} & \textbf{Dirección IP} \\ 
		\hline 
		pc2 &  &   \\ 
		\hline 
		pc4 &  &  \\ 
		\hline 
		r5 &  &  \\ 
		\hline 
	\end{tabular} 
    \egroup
    \medskip
    
\begin{enumerate}[resume]
	\item Partiendo de estado inicial, antes de haber ejecutado ningún comando \texttt{ping}, 
	comprueba el estado de las cachés de ARP. Saca una captura de la terminal de cada máquina 
	(\texttt{pc2}, \texttt{pc4} y \texttt{r5}) con el comando que has usado y el resultado.
	
	\item Arranca \texttt{tcpdump} para capturar el tráfico en \texttt{pc4} y en \texttt{r5} (en su 
	interfaz \texttt{eth2}). Realiza un único \texttt{ping} desde \texttt{pc2} a \texttt{pc4}. Guarda 
	las capturas en los ficheros \texttt{cap-pc4-xxx.cap} y \texttt{cap-r5-xxx.cap}, donde \texttt{xxx}
	son tus iniciales (iguales que las que usaste en los ejercicios del Bloque II). Toma una captura 
	de pantalla del terminal en \texttt{pc4} y \texttt{r5} donde se vea el comando \texttt{tcpdump} 
	lanzado.
	
	\item Comprueba de nuevo el estado de las cachés de ARP y saca las capturas de la terminal 
	de cada máquina (\texttt{pc2}, \texttt{pc4} y \texttt{r5}). Compáralas con lo obtenido en la \textbf{pregunta 8} 
	y explica qué ha ocurrido.
	
	\item Abre las dos capturas en Wireshark, toma un pantallazo de cada una y adjúntalos como 
	respuesta a este apartado.
	
	\item Explica cada una de las tramas que aparecen en la captura tomada en \texttt{r5}, indicando 
	el protocolo, quién la envía, quien la recibe y el propósito.
\end{enumerate}

%%%---------------------------------------------- 

\section{UDP}
\label{sec:udp}

\begin{enumerate}[resume]
	\item Partimos del mismo escenario anterior, en el que hay conectividad entre \texttt{pc2} y 
	\texttt{pc4} a través de \texttt{r5}. Arranca sólo \texttt{pc2}, \texttt{pc4} y \texttt{r5}. Queremos enviar 
	el mensaje \texttt{"Hola, soy xxx"}, donde \texttt{xxx} son tus iniciales (igual que en los ejercicios 
	previos) desde \texttt{pc2} a \texttt{pc4}, mediante el protocolo UDP. Para ello usaremos el 
	comando \texttt{nc} (\texttt{netcat}, que ya se vio en las prácticas). 
	
	Como primer paso arranca el comando \texttt{tcpdump} en \texttt{r5} para que se realice una 
	captura del tráfico que pasa por su interfaz \texttt{eth0}. Guárdalo en el fichero \texttt{udp-xxx.cap}, 
	donde \texttt{xxx} son tus iniciales. Lanza \texttt{tcpump}, toma una captura del terminal y 
	adjúntala como respuesta a esta pregunta.
	
	\item Para recibir el mensaje en \texttt{pc4} deberás lanzar \texttt{nc} en modo escucha desde un 
	puerto que comience por el número \texttt{40} y termine con los dos primeros dígitos de tu DNI. Así, 
	si tu dni es \texttt{32531021T}, el puerto será \texttt{4032}. Desde \texttt{pc2} lanza el comando 
	\texttt{nc} en modo cliente usando un puerto origen que comience por \texttt{20} y termine con 
	los primeros dígitos de tu DNI. Como puerto destino usa el mismo que has empleado en el comando 
	usado en \texttt{pc4}. Una vez lanzados ambos comandos, toma una captura de pantalla de los dos 
	termines y adjúntalos como respuesta a esta pregunta. Todavía no hemos enviado el mensaje de texto.
	
	\item Desde \texttt{pc2} escribe el mensaje \texttt{"Hola, soy xxx"}, donde \texttt{xxx} son tus iniciales. 
	Interrumpe el comando \texttt{nc} del ordenador \texttt{pc2}. Interrumpte la captura en \texttt{r5}, 
	y adjunta pantallazos de los terminales de \texttt{pc2} y \texttt{pc4}. ¿Se ha recibido en \texttt{pc4} 
	el mensaje enviado?
	
	\item Abre la captura con Wireshark. Saca un pantallazo de todas las tramas recibidas y adjúntalo 
	como respuesta a esta pregunta.
	
	\item Indica de entre todas las tramas capturadas cuáles son de apertura de la conexión, y cuáles 
	de cierre.
	
	\item Busca la trama que envía los datos. Toma una captura de wiresark en la que se puedan ver 
	los datos enviados de \texttt{pc2} a \texttt{pc4}.
	
	\item Repite el experimento, pero ahora el comando \texttt{nc} del terminal \texttt{pc2} debe 
	conectarse al puerto destino \texttt{8000}. Comienza una captura nueva, en el fichero 
	\texttt{udp2-xxx.cap} (donde \texttt{xxx} son tus iniciales), envía el mensaje \texttt{"Probando"} 
	(desde \texttt{pc2} a \texttt{pc4}) y cierra la captura. Abre el fichero con Wireshark y adjunta en
	tu respuesta un	pantallazo con las tramas capturadas.
	
	\item Explica cada una de las tramas, y qué es lo que ha sucedido. ¿Ha recibido \texttt{pc4} el mensaje?
	
\end{enumerate}

%%%---------------------------------------------- 

\section{TCP}
\label{sec:tcp}

\begin{enumerate}[resume]
	\item Partimos del mismo escenario anterior. Queremos enviar el mismo mensaje desde \texttt{pc2} a 
	\texttt{pc4}:  \texttt{"Hola, soy xxx"}, donde \texttt{xxx} son tus iniciales. Pero ahora usaremos el 
	protocolo TCP. Arranca el comando \texttt{tcpdump} en \texttt{r5} para realizar una captura en su 
	interfaz \texttt{eth0}. Guárdalo en el fichero \texttt{tcp-xxx.cap}, donde \texttt{xxx} son tus iniciales. 
	Lanza \texttt{tcpdump} y toma una captura del terminal. Adjunta la captura como respuesta a esta 
	pregunta.
	
	\item Lanza \texttt{nc} en \texttt{pc4} en modo escucha, desde un puerto que comience por el número 
	\texttt{40} y termine con los dos primeros dígitos de tu DNI (igual que en el apartado de UDP). Desde 
	\texttt{pc2} lanza el comando \texttt{nc} en modo cliente un puerto origen que comience por \texttt{20} 
	y termine con los primeros dígitos de tu DNI. Como puerto destino usa el mismo que has empleado 
	en el comando usado en \texttt{pc4}. Una vez lanzados ambos comandos, toma una captura de pantalla 
	de los dos terminales y adjúntalas como respuesta a esta pregunta. Todavía no hemos enviado el 
	mensaje de texto.
	
	\item Desde \texttt{pc2} escribe el mensaje \texttt{"Hola, soy xxx"}, donde \texttt{xxx} son tus iniciales. 
	Interrumpe el comando \texttt{nc} del ordenador \texttt{pc2}. Interrumpe la captura en \texttt{r5}, y 
	adjunta pantallazos de los terminales de \texttt{pc2} y \texttt{pc4}. ¿Se ha recibido en \texttt{pc4} 
	el mensaje enviado?
	
	\item Abre la captura con Wireshark. Saca un pantallazo de todas las tramas recibidas y adjúntalo 
	como respuesta a esta pregunta.
	
	\item Indica de entre todas las tramas capturadas cuáles son de apertura de la conexión, y cuáles de 
	cierre.
	
	\item Explica cada una de las tramas y qué es lo que ha sucedido.
	
\end{enumerate}
\bigskip

%%%---------------------------------------------- 

% Final glyph to end document
\begin{center}
	{\LARGE\adforn{21}\quad\adforn{33}\quad\adforn{49}}
\end{center}

\end{document}

